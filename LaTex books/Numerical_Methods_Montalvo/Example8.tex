\subsection{Example Problems}

\begin{enumerate}

\item Write a routine that will use Polynomial Least Squares regression. The
routine will have a function header like this
\begin{framed}
\textcolor{blue}{function} coeff = myregression(Y,X,N)
\end{framed}
\noindent where Y is the vector of sampled point $y_i$ and X is a vector of
independent points $x_i$. N is the order of the fit. When N = 0, the
fit is zero order, when N = 1 the fit is linear and when N = 2 the fit
is quadratic. For the data below plot the data given on a figure and
compute the least squares fit. Note, you will have to plot the data to
figure out which order to pick. Pick an order that reduces to the
least square error to a minimum. Once you have the fit, plot the fit on
the graph and put the least square error in the title. 
\ \\

In modeling an oil reservoir, it may be necessary to find a
relationship between the equilibrium constant of a reaction and the
pressure at constant temperature. 

\begin{equation}\nonumber
\begin{matrix}
K-value & Pressure \\
7.5 & 0.635\\
5.58 & 1.035\\
4.35 & 1.435\\
3.55 & 1.835\\
2.97 & 2.235\\
2.53 & 2.635\\
2.2 & 3.035\\
1.93 & 3.435\\
1.7 & 3.835\\
1.46 & 4.235\\
1.28 & 4.635\\
1.11 & 5.035\\
1.0 & 4.435\\
\end{matrix}
\end{equation}

\item Write a routine that will compute a 2nd-order planar fit. That
  is, assume the form

  \begin{equation}\nonumber
    z = a_0 + a_1sin(2x) + a_2sin(2y)
  \end{equation}

  Use Gauss' equation to solve for the
  coefficients and create a mesh of the solution. The data will be
  provided in a text file with columns x,y and z. First plot the data
  using blue stars. These can be used to solve for the coefficients
  $a_0,a_1$ and $a_2$. To create a mesh use the following code. 

  \begin{framed}
    xest = linspace(-pi,pi,100); 

    yest = linspace(-pi,pi,100);

    [xx,yy] = meshgrid(xest,yest);

    zz = a0 + a1*sin(2*xx) + a2*sin(2*yy);
    
    mesh(xx,yy,zz)
  \end{framed}
  Your results should have 1 graph with the provided data and the
  mesh. You need to also include your coefficients in your answer as well.

\item The data below describes the growth of a population following a
  logistical model. This system is non-linear however the equation can
  be converted to a linear system.

  \begin{equation}\nonumber
    \begin{matrix}
    y = \frac{1}{1+e^{ax+b}} \\
    z = 1/y - 1 \\
    zz = ln(z)
    \end{matrix}
  \end{equation}

  Use the equations above to solve for the coefficients a and b and
  plot the solution on the same graph.

  \begin{equation}\nonumber
    \begin{matrix}
      X & Y \\
      -1.0 & 0.05 \\
      -0.8 & 0.08 \\
      -0.6 & 0.14 \\
      -0.4 & 0.23 \\
      -0.2 & 0.35 \\ 
      0.0 & 0.50 \\
      0.2 & 0.65 \\
      0.4 & 0.77 \\
      0.6 & 0.86 \\
      0.8 & 0.92 \\
      1.0 & 0.95
    \end{matrix}
  \end{equation}

\end{enumerate}


