\subsection{Example Problems}

\begin{enumerate}

\item Create a function that has the following function header. 

\textcolor{blue}{function} [mass,volume,weight] = myellipse(name, major\_axis, minor\_axis,
transverse\_axis,density)

This function will compute the volume, mass, and weight(on earth) of
an ellipse. Furthermore, place an if statement in the function that
will display(using the disp() function) ``The ellipse: (blank) is too
heavy'', if the ellipse if heavier than 5 kg. (Use SI units for
everything) 
\ \\

\noindent Experiment with at least 3 different types of ellipses to
ensure your function is working properly. Include your inputs and
outputs in your document. 

\item Create a function that adds up the entries of a vector. Here is
  an example function header

\textcolor{blue}{function} s = mysum(x)

Remember you may not use the function sum() in MATLAB however you may
use it to check your answers. Test this on 3 example vectors to
ensure your functions are working properly.  

\item Create a function that finds the minimum value in a vector. Here
  is an example function header

\textcolor{blue}{function} m = mymin(x)

Remember you may not use the functions min() or find() in MATLAB
however you may use them to check your answers. Test this on 3 example
vectors to ensure your functions are working properly. 

\item Use the method of loops to compute the fibonacci sequence until
the number of digits in the sequence is more than 2 (so 100 or
bigger). Save the sequence into a vector as you move through the
sequence and include it in your document. The fibonacci sequence can be
written using the equation below. 

\begin{equation}
\lambda_{i+1} = \lambda_{i} + \lambda_{i-1}
\end{equation}

To start the sequence assume that $\lambda_{-1} = 0$ and $\lambda_{0}
= 1$. The first three steps of the sequence are shown below.

\begin{equation}
\begin{matrix}
\lambda_{1} = \lambda_{0} + \lambda_{-1} = 1 + 0 = 1 \\
\lambda_{2} = \lambda_{1} + \lambda_{0} = 1 + 1 = 2 \\
\lambda_{3} = \lambda_{2} + \lambda_{1} = 2 + 1 = 3 \\
\end{matrix}
\end{equation}

\item Assume I create a structure that chracterizes the properties of an
ellipse. Assume that the current fields of the structure are:
name, major\_axis, minor\_axis, transverse\_axis, and density. Create a function that will
compute the volume, mass, and weight(on earth) of all ellipses in the
structure and add it to a field called by a similar name and output it to the
workspace. Furthermore, place an if statement in the function that
will display ``The ellipse: (blank) is too heavy'', if the ellipse if
heavier than 5 kg.
\ \\

\noindent Use the following lines of code to test your function (Notice the
difference in units!)
\ \\

ellipses(1).name = 'Kids ball';

ellipses(1).major\_axis = 21.59; \%cm

ellipses(1).minor\_axis = 21.59; \%cm

ellipses(1).transverse\_axis = 21.59; \%cm

ellipses(1).density = 3.0; \%$g/cm^3$

ellipses(2).name = 'Hindenburg'; 

ellipses(2).major\_axis = 803.8/3.28; \%meters

ellipses(2).minor\_axis = 135.1/3.28; \%meters

ellipses(2).transverse\_axis = 135.1/3.28; \%meters

ellipses(2).density = 1.2; \%$kg/m^3$
\ \\

\item The following website has some useful information about the
orbit of the earth:
\ \\

``http://en.wikipedia.org/wiki/Earth's\_orbit''
\ \\

Assuming the Sun is located at 0,0 and the orbit of the earth
is flat (2D). Plot the orbit of the earth around the sun. Where is the
earth right now? Plot a large blue circle where the earth currently
is. Assume today is September 3rd, 2014. Furthermore, plot a large
yellow circle where the sun is.
\ \\

\item Create a function that will plot the trajectory of a
sphere. Assume the inputs are the angle from the horizontal in radians
and the speed in m/s. Neglect aerodynamic drag and assume the sphere is thrown on
earth. Furthermore, assume the sphere is launched at time t = 0 and
compute the time the ball hits the ground. Experiment with this
function and determine the optimal angle from the horizontal that will
throw the sphere the farthest assuming a constant speed. Then compute
the optimal angle that will give the ball the most hang time for a
given speed. Is it different than the farthest distance? Does this
make sense?
\ \\

\end{enumerate}

\subsection{Project}

By now you must have solved some pretty difficult problems in your
other classes. Every engineering problem can be cast into the form  \\
\ \\
dependent variable = function( independent variable,parameters)\\
\ \\
The example of computing the terminal
velocity of a cat is a perfect example of this. Your task it to team
up with 2 more individuals and write a MATLAB code to solve three
problems from your other classes. \\
\ \\
The first step is to identify three problems that can be solved by
hand but have parameters and independent variables that change. In the
cat falling example, our parameters were the area and weight of the
cat. We could perhaps keep the size of the cat fixed but vary the
weight. This MATLAB code can then be used to compute the terminal
velocity of the cat just by changing the weight of the cat.  \\
\ \\
Your deliverable for this assignment will be to write a report
detailing your 3 functions. The sections included in your report will
be the following \\

\begin{enumerate}
\item{{\bf Introduction}} 
Explain what the problems are. Why do we care? Why
is this important? Give some background on this type of problem. 
\item{{\bf Mathematical Model}}
Explain the theory on how these problems are
solved. Include equations in your report. Do not screenshot equations
or just type them in. You are engineers. It's time to learn how to use
Equation Editor. Finally, include all pertinent data required to run
your code. Are there fixed parameters that do not vary? Include them
in this section. 
\item{{\bf Results}}
Explain your inputs to your code and your outputs. Do
not copy and paste MATLAB output. Write your results in normal
english. For example, "When the weight of the cat is 5 lbs the
terminal velocity is 50 ft/s. If the weight of the cat is increased to
10 lbs the terminal velocity of the cat is 80 ft/s".  
\item{{\bf Appendix MATLAB Code}}
Copy and paste your MATLAB code. This is
the only place the word MATLAB should be. No supporting text required,
simply copy and paste your code into this section. 
\end{enumerate}
