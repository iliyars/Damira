\subsection{Example Problems}

\begin{enumerate}

\item Use your script from last week to plot the fibonacci sequence as
  a function of iteration number. Include the figure in your
  word document however make sure your graph looks pretty (i.e. label
  your axes, add a grid, etc). The fibonacci sequence can be written
  using the equation below.  

  \begin{equation}
    \lambda_{n+2} = \lambda_{n+1} + \lambda_{n}
  \end{equation}
  
  To start the sequence assume that $\lambda_{1} = 0$ and $\lambda_{2}
  = 1$. 

\item Simulate the system below for three different values of $\Delta
  x$. Use 1,0.5 and 0.1. However simulate the system until $x_{n+1} ==
  5$. Let $x(1) = 0$ and $y(1) = 1$.

\begin{equation}
\begin{matrix}
  y_{n+1} = (1-2\Delta x)y_{n} \\ x_{n+1} = x_{n} + \Delta x
\end{matrix}
\end{equation}

  Create a figure and plot y on the y-axis and x on the
  x-axis. Include all three lines on your figure. Remember to plot
  each line in a different color and for this example you will need to
  add a legend. Again make your plot look nice and include it in your
  document. 

\item Make a 3-Dimensional object of your choosing using mesh just
  like I did in my youtube video. You can make an ellipse, or a bowl,
  or a cup, or a ball, a pyramid, etc. Any 3-dimensional
  object. Again, make the graph look nice and include your figure in
  your document. 

\item Using your function from problem 1, Homework 3 edit the function to
plot the ellipsoid that is read in using mesh(). That is, in your for
loop create a figure and mesh an ellipsoid. Label your axes, make the
background white, create a title with the name of the ellipse and set
the viewport to [-27,30]. Use the view() command. The run script will
be the same as problem 1 from the previous homework thus your function
should create two ellipses in this example.
\ \\

\item I have uploaded an excel spreadsheet with information about all
planets in the solar system including Pluto (As far as I'm
concerned it's still a planet). Edit your function in problem 2
Homework 3 to read in the excel spreadsheet and loop through all
planets and plot the orbits of all planets. You should end up
generating a plot with 9 orbits. There is no need to plot the location
of the planets in this example. This function will have no outputs or
inputs. It merely needs to generate a plot.
\ \\

\item Using your function from problem 3 Homework 3 edit the function to
make a movie of the ball traveling through the air. Name the movie
file M for my convenience and make it the output of the
function. There should only be one output (the movie file M).
\ \\

\end{enumerate}
