\section{Aerospace Equations of Motion}

The translational equations of motion of
the vehicle are written using inertial coordinates using the center
of the Earth as a fixed point. For up to cis-lunar orbits this is typically
a good approximation for first order analysis. The vehicle is
assumed to be a rigid body using quaternions to parameterize
orientation. 

\subsection{Translational Equations of Motion}

The translational equations of motion of satellites are fairly simple given
that everything is written in the inertial frame. The position vector
of the vehicle is $\vec{r} = [x,y,z]^T$ and the velocity is
$\vec{V}_{B/I}=[\dot{x},\dot{y},\dot{z}]^T$. The acceleration of
the vehicle is found by summing the total forces on the body
and dividing by the mass of the vehicle. In the equation below
$N_\Earth$ is the number of planetary bodies acting on the vehicle
while $\vec{F}_P$ is the force imparted by thrusters. 
\begin{equation}
  \vec{a}_{B/I} = \frac{1}{m_s}\left(\sum\limits_{i=1}^{N_\Earth}F_i + \vec{F}_P \right)
\end{equation}
Note that for a spacecraft the magnitude of the gravitational
acceleration vector is on the order of $\pm 10~m/s^2$. Sources point
to solar radiation pressure being on the order of $4.5~\mu Pa$
\cite{Radiation}. For a 1U CubeSat (10~cm x 10~cm) 
the force would be equal to $0.45 mN$. A 1U CubeSat has a nominal
mass of 1 kg which would accelerate the CubeSat on the order of
$0.45 mm/s^2$, which is considerably less than gravitational
acceleration. Furthermore, using the standard aerodynamic drag
equation ($0.5\rho V^2 S C_D$), where conservative estimates are used,
the aerodynamic force at 600 km above the Earth's surface would be
about $3.0~nN$ \cite{AndersonD}. This assumes a density equal to $1.03
\times 10^{-14}~kg/m^3$, a velocity equal to $7.56~km/s$,
and a drag coefficient equal to 1.0 \cite{Density_Model}. A force this
small would impart an acceleration of about $3.0~nm/s^2$ which is also
considerably less than gravitational acceleration. These forces cannot be
neglected for longer missions but can be ignored where
appropriate. For an aircraft and quadcopter the equations of motion
are typically written in the body frame. As such the derivative
transport theorem is used and the translational equations of motion
are written as the following.
\begin{equation}\label{e:uvwdot} 
\begin{Bmatrix} \dot{u} \\ \dot{v} \\ \dot{w} \end{Bmatrix} = 
\frac{1}m \begin{Bmatrix} X\\Y\\Z\end{Bmatrix}-\begin{bmatrix} 0 & -r
& q \\ r & 0 & -p \\ -q & p & 0 \end{bmatrix} \begin{Bmatrix} u\\v\\w\end{Bmatrix}
\end{equation}

\subsection{Reaction Wheel Model}

The reaction wheel model must be included before the attitude dynamics
because they directly affect the inertia of the vehicle. There are
three reaction wheels on this vehicle and each one has it's own
angular velocity $\omega_{Ri}$ and angular acceleration
$\alpha_{Ri}$. The inertia of each reaction wheel is first written
about the center of mass of the reaction wheel and is given by the
equation below where the reaction wheel is modeled as a disk with
finite radius $(r_{RW})$ and height $(h_{RW})$. The subscript $R$ is used to
denote that this inertia matrix is about the center of mass of the
reaction wheel while the super script $R$ is used to denote the frame
of reference. 
\begin{equation}
  I^{R}_{Ri} = \begin{bmatrix} m_{R}r^2/2 & 0 & 0 \\ 0 & (m_R/12)(3r_{RW}^2+h_{RW}^2) &
    0 \\ 0 & 0 & (m_R/12)(3r_{RW}^2+h_{RW}^2) \end{bmatrix}
\end{equation}
In order to rotate the inertia matrix into the vehicle body frame of
reference an axis of reaction wheel rotation is used. The vector
$\hat{n}_{Ri}$ is used to denote the axis about which the reaction
wheel rotates. Euler Angles $\theta_{Ri}$ and $\psi_{Ri}$ can be
extracted from this unit vector as discussed previously in Section \ref{s:Euler_Angles}. The rotation
matrix ${\bf T_{Ri}}(0,\theta_{Ri},\psi_{Ri})$ can then be 
generated using equation \ref{e:TIB}. This matrix can then be used to
compute the inertia of the reaction wheel in the vehicle body frame.
\begin{equation}
  I^{B}_{Ri} = {\bf T}^T_{Ri}I^{R}_R{\bf T}_{Ri}
\end{equation}
The parallel axis theorem can then be used to shift the inertias to
the center of mass of the vehicle where the subscript $RB$ denotes
the reaction wheel inertia taken about the center of mass of the
vehicle. 
\begin{equation}
  I^{B}_{RBi} = I^{B}_{Ri} + m_{Ri}{\bf S}(\vec{r}_{Ri}){\bf
    S}(\vec{r}_{Ri})^T
\end{equation}
The vector $\vec{r}_{Ri}$ is the distance from the center of mass of
the vehicle to the center of mass of the reaction wheel in the
vehicle body reference frame. The total inertia of the entire
vehicle-reaction wheel system is then just a sum of all the reaction wheel inertias.
\begin{equation}
   I_S = I_B + \sum\limits_{i=1}^3 I^{B}_{RBi}
\end{equation}
The total angular momentum of the vehicle is then equal to the following
equation where $\vec{\omega}_{B/I}$ is the angular velocity of the
vehicle. 
\begin{equation}
  \vec{H}_S = I_B\vec{\omega}_{B/I} + \sum\limits_{i=1}^3
  I^{B}_{Ri}\omega_{Ri}\hat{n}_{Ri}
\end{equation}
In a similar fashion, the total torque placed on the vehicle is
equal to the following
\begin{equation}
  \vec{M}_{R} = \sum\limits_{i=1}^3 I^{B}_{Ri}\alpha_{Ri}\hat{n}_{Ri}
\end{equation}
It is typically assumed that the angular acceleration of each reaction
wheel can be directly controlled. However, as the reaction wheel
angular velocity increases, the maximum angular acceleration allowed
begins to decrease. Once the reaction wheel reaches its angular
velocity limits, the angular acceleration possible drops to zero. This
is called reaction wheel saturation and must be dealt with using a
method called momentum dumping.

\subsection{Attitude Equations of Motion}

The attitude equations of motion are formulated assuming the vehicle
can rotate about three axes. The derivative of
angular velocity is found by equating the derivative of angular momentum
to the total moments placed on the vehicle while reaction wheel
torques from the vehicle are added.(\cite{etkins}).
\begin{equation}\label{e:pqrdot}
  \dot{\vec{\omega}}_{B/I} = I_S^{-1}\left( \vec{M}_P + \vec{M}_{M} +
  \vec{M}_R - {\bf S}(\vec{\omega}_{B/I}) \vec{H}_{S} - \dot{I}_S\vec{\omega}_{B/I}\right)
\end{equation}
The applied moments use subscripts $(P)$ for propulsion, $(M)$ for
magnetorquers, and $(R)$ for reaction wheels. The term $\dot{I}_S$ is the change in
inertia in the body frame caused by deployment of solar panels and/or
antenna. Also, recall that $\vec{H}_S$ is the total angular momentum
of the entire vehicle including the reaction wheels if present. For
aircraft the rotational dynamic equation can be found as
\begin{equation}\label{e:pqrdot} 
\begin{Bmatrix} \dot{p} \\ \dot{q} \\ \dot{r} \end{Bmatrix} = {\bf I}_C^{-1}\left(
\begin{Bmatrix} L\\M\\N\end{Bmatrix}-\begin{bmatrix} 0 & -r
& q \\ r & 0 & -p \\ -q & p & 0 \end{bmatrix} {\bf I}_C\begin{Bmatrix} p\\q\\r\end{Bmatrix}\right)
\end{equation}
