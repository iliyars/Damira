\section{Numerical Integration Techniques}

\subsection{Euler's Method}

The equations of motion presented in this text as well as any
differential equation can be integrated using Euler's method
which is a crude first order method to approximate the time series
solution \cite{Chapra_MEANALYSIS}. Note that this method is prone to a 
significant amount of instability unless the timestep is very small. 
\begin{equation}
  \begin{matrix}
    \vec{x}_{k+1} = \vec{x}_k + \dot{\vec{x}}(t_k,\vec{x}_k) \Delta t \\
    \dot{\vec{x}}(t_k,\vec{x}_k) = \vec{f}(\vec{x}_k) + \vec{g}(\vec{x}_k)\vec{u}_k
  \end{matrix}
\end{equation}

\subsection{Runge-Kutta-4}

The RK4 algorithm is the standard in numerical integration and is
given in the equation below \cite{Chapra_MEANALYSIS}. The derivative
of the quaternions is the same in RK4 as it is in Euler's method. This
method is superior to RK4 in that it will converge faster as a
function of timestep.
\begin{equation}
  \begin{matrix}
    \vec{k}_1 = \dot{\vec{x}}(t_k,\vec{x}_k)\\
    \vec{k}_2 = \dot{\vec{x}}(t_k+\Delta t/2,\vec{x}_k+\vec{k}_1\Delta t/2)\\
    \vec{k}_3 = \dot{\vec{x}}(t_k+\Delta t/2,\vec{x}_k+\vec{k}_2\Delta t/2)\\
    \vec{k}_4 = \dot{\vec{x}}(t_k+\Delta t,\vec{x}_k+\vec{k}_3\Delta t)\\
    \vec{k} = \frac{1}{6}(\vec{k}_1 + 2\vec{k}_2 + 2\vec{k}_3 + \vec{k}_4)\\
    \vec{x}_{k+1} = \vec{x}_k + \vec{k} \Delta t \\
  \end{matrix}
\end{equation}

\subsection{Discrete Dynamics}

It is often useful for modern computers to write the equations of
motion in discrete form....
