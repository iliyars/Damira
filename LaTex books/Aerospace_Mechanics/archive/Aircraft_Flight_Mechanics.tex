%\documentclass[11pt,a4paper,twoside,openright]{book}
\documentclass{article}
\usepackage{amsmath}
\usepackage{geometry}
\usepackage{fullpage}
\usepackage[usenames,dvipsnames]{color}
%\usepackage{amsmath}
\usepackage{graphicx}
\usepackage{float}
\usepackage{datetime}
\usepackage{natbib} %REMOVE THIS IF YOU DON'T WANT BRACKETS ON YOUR PAPER
\usepackage{booktabs}
\usepackage{multirow}
\usepackage{hyperref}
\usepackage{enumitem}
\usepackage{wrapfig}		%					allows wrapping text around figures
\usepackage{multicol}
%\usepackage{cite}
\usepackage[font=normalsize,labelfont=bf]{caption}
\usepackage{titlesec}
\titleformat{\section}[block]{\large\bfseries}{\thesection}{1em}{}
\usepackage{framed}
\usepackage{setspace}
\def\beq{\begin{equation}}
\def\eeq{\end{equation}}
\def\beqn{\begin{matrix}}
\def\eeqn{\end{matrix}}

\setlength{\marginparwidth}{0pt}

% Define commands to assure consistent treatment throughout document
\newcommand{\eqnref}[1]{(\ref{#1})}
 \newcommand{\class}[1]{\texttt{#1}}
 \newcommand{\package}[1]{\texttt{#1}}
 \newcommand{\file}[1]{\texttt{#1}}
 \newcommand{\BibTeX}{\textsc{Bib}\TeX}
\renewcommand{\thesection}{\Roman{section}.}
\renewcommand{\thesubsection}{\arabic{subsection}.}
\renewcommand{\thesubsubsection}{\alph{subsubsection}.}

\begin{document}

\begin{center}
\begin{LARGE}{\bf Aircraft Flight Mechanics}\end{LARGE}\\
\large
\vspace{22 mm}
\begin{singlespace}
   A Brief Textbook Presented to the \\ 
   Student Body of the University of South Alabama \\
\vspace{22 mm}
   by\\
\vspace{22 mm}
  Carlos Montalvo \\
\vspace{22 mm}
\end{singlespace}
{\itshape University of South Alabama}\\
{\bf Copyright $\copyright$ Carlos Jos\'{e} Montalvo}
\\ Last Update: \today
\end{center}

\newpage

\section*{Manuscript Changes}

\begin{enumerate}[itemsep=-5pt]
\item December 10th, 2020 - Moved manuscript to public Github repo
\item December 30th, 2020 - Added datetime to title page. Added
  section headers to RC aircraft design. Wrote text all the way
  through the airfoil selection. Added a references section.
\item December 31st, 2020 - Finished RC aircraft section.
\end{enumerate}

\newpage

\section*{Changes Needed}

\begin{enumerate}[itemsep=-5pt]
\item Need to finish section on how to design an RC aircraft
\item It would be nice to include some plots on RC aircraft design
  sections
\item I think an example RC aircraft design example would be good
\item I think a section on abbreviations would be good.
\end{enumerate}

\tableofcontents

\newpage

\section{Particle Dynamics}
For this formulation we start with Newton's Second Law with no
approximations\cite{multibody}.
\begin{equation}
\sum\limits_{i=0}^N \vec{F}_{ji} = \frac{d\vec{p}_j}{dt}
\end{equation}
where $\vec{p}_j$ is the momentum of a particle. $\vec{F}_{ji}$ is a
force on the particle. The statement above states that sum of all
forces on a particle is equal to the time rate of change of
momentum. 

\subsection{Linear Dynamics for Systems of Particles}
If two particles are then considered the equation can
be written for both particles.
\begin{equation}
\begin{matrix}
\sum\limits_{i=0}^N \vec{F}_{1i} + \vec{f}_{12} = \frac{d\vec{p}_1}{dt} &
\sum\limits_{i=0}^N \vec{F}_{2i} + \vec{f}_{21} = \frac{d\vec{p}_2}{dt} 
\end{matrix}
\end{equation}
Note that the forces $\vec{f}_{12}$ and $\vec{f}_{21}$ are internal forces
experienced by each particle exerted on each other since they are
rigidly connected. Newton's Third Law states that for every action
there is an equal and opposite reaction. That is, $\vec{f}_{12} = -\vec{f}_{21}$. Thus, if both equations are added the following equation is
created
\begin{equation}
\sum\limits_{j=0}^P \sum\limits_{i=0}^N \vec{F}_{ji} =
\sum\limits_{j=0}^P \frac{d\vec{p}_j}{dt}
\end{equation}
where P is the number of particles. Typically the double summation
in F is written just as $\vec{F}$. 

\subsection{Rotational Dynamics for Systems of Particles}

Note that by construction, a system of particles rigidly connected can
now rotate about a center point. The center of mass of a system of
particles can be defined using the relationship below
\begin{equation}
\vec{r}_C = \frac{1}{m}\sum\limits_{j=0}^P m_j\vec{r}_{j}
\end{equation}
where
\begin{equation}
m = \sum\limits_{j=0}^P m_j
\end{equation}
This vector can then be used to create rotational dynamics starting
with the linear dynamics.
\begin{equation}
\sum\limits_{j=0}^P \sum\limits_{i=0}^N {\bf S}(\vec{r}_{Cj}) \vec{F}_{ji}  = \vec{M}_C = \sum\limits_{j=0}^P {\bf
  S}(\vec{r}_{Cj}) \frac{d\vec{p}_j}{dt}
\end{equation}
where ${\bf S}(\vec{r}_{Cj})$ is the skew symmetric matrix of the
vector from the center of mass to the jth particle which results in a cross product.
\section{Rigid Bodies}
At this point, many assumptions are made about the
system of particles.
\begin{enumerate}
\item The mass of each particle or rigid body is constant.
\item The Earth is assumed to be flat and not rotating. This allows
  the Earth to be used as an inertial frame.
\item The rigid body is not flexible and does not change shape. That
  is, the time rate of change of the magnitude of a vector
  $\vec{r}_{PQ}$ is zero for any arbitrary points P and Q attached to
  the rigid body.
\end{enumerate}
\subsection{Linear Dynamics}
Using all of these simplifications, the momentum term on the right can
be simplified to 
\begin{equation}
\sum\limits_{j=0}^P \vec{p}_j = m \vec{v}_{C/I}
\end{equation}
The derivation of the term above starts by deriving the position of
the center of mass as the following equation.
\begin{equation}
\vec{r}_{j} = \vec{r}_C + \vec{r}_{Cj}
\end{equation}
Taking one derivative results in the following equation
\begin{equation}
\vec{v}_{j/I} = \vec{v}_{C/I} + \frac{{}^Bd \vec{r}_{Cj}}{dt} +
{\bf S}(\vec{\omega}_{B/I}) \vec{r}_{Cj}
\end{equation}
where ${\bf S}(\vec{\omega}_{B/I})$ is the skew symmetric matrix of the
angular velocity vector which results in a cross product. This
equation comes from the derivative transport theorem. Since the body
is a rigid body the term $\frac{{}^Bd \vec{r}_{Cj}}{dt}=0$ resulting
in the equation below
\begin{equation}
\vec{v}_{j/I} = \vec{v}_{C/I} + {\bf S}(\vec{\omega}_{B/I}) \vec{r}_{Cj}
\end{equation}
which any dynamicist knows as the equation for two points fixed on a
rigid body. This equation can then be substituted into the equation
for momentum such that.
\begin{equation}
\sum\limits_{j=0}^P \vec{p}_j =  \sum\limits_{j=0}^P m_j \left(\vec{v}_{C/I}
+ {\bf S}(\vec{\omega}_{B/I}) \vec{r}_{Cj}\right)
\end{equation}
The first term reduces to 
\begin{equation}
\sum\limits_{j=0}^P m_j \vec{v}_{C/I} =  \vec{v}_{C/I}
\sum\limits_{j=0}^P m_j =  m \vec{v}_{C/I} 
\end{equation}
the second term reduces to zero since the sum of all particles from
the center of mass is by definition the center of mass and thus zero.
\begin{equation}
\sum\limits_{j=0}^P {\bf S}(\vec{\omega}_{B/I}) m_j\vec{r}_{Cj} =
{\bf S}(\vec{\omega}_{B/I})\sum\limits_{j=0}^P m_j\vec{r}_{Cj} = 0
\end{equation}
Plugging this result for momentum into Newton's equation of motion
yields. This is typically called Newton-Euler equations of motion.
\begin{equation}
\vec{F}_C = m \left(\frac{{}^Bd \vec{v}_{C/I}}{dt} +
{\bf S}(\vec{\omega})_{B/I} \vec{v}_{C/I} \right)
\end{equation}
\subsection{Rotational Dynamics}
Plugging in the expression for two points fixed on a rigid body
results in a much different expression. First let's expand the
rotational dynamic equations of particles using the assumptions made
for a rigid body.
\begin{equation}
\vec{M}_C = \frac{d}{dt}\sum\limits_{j=0}^P {\bf S}(\vec{r}_{Cj}) m_j\vec{v}_{j/I}
\end{equation}
Then the equation of two points fixed on a rigid body can be
introduced to obtain the following equation
\begin{equation}
\vec{M}_C = \frac{d}{dt}\sum\limits_{j=0}^P {\bf S}(\vec{r}_{Cj}) m_j\left(\vec{v}_{C/I} + {\bf S}(\vec{\omega}_{B/I}) \vec{r}_{Cj}\right)
\end{equation}
expanding this into two terms yields
\begin{equation}
\vec{M}_C =  \frac{d}{dt}\left(\sum\limits_{j=0}^P m_j{\bf S}(\vec{r}_{Cj}) {\bf S}(\vec{\omega}_{B/I}) \vec{r}_{Cj} + \sum\limits_{j=0}^P {\bf S}(\vec{r}_{Cj})m_j\vec{v}_{C/I}\right)
\end{equation}
To simplify this further a useful equality is used for cross
products. That is ${\bf S}(\vec{a})\vec{b}=-{\bf
  S}(\vec{b})\vec{a}$. The equation above then changes to
\begin{equation}
\vec{M}_C =  \frac{d}{dt}\left(\left(-\sum\limits_{j=0}^P m_j{\bf S}(\vec{r}_{Cj}){\bf
  S}(\vec{r}_{Cj})\right)\vec{\omega}_{B/I} - {\bf S}(\vec{v}_{C/I})
\sum\limits_{j=0}^P \vec{r}_{Cj}m_j\right)
\end{equation}
Notice, that parentheses were placed around the first term to isolate
the angular velocity. This is because the angular velocity is constant
across the system of particles. The term on the right has also been
altered slightly to isolate the fact that the velocity of the center
of mass is independent of the system of particles. With the equation
in this form it is easy to see that the term on the right is zero
because it is the definition of the center of mass. The equation then
reduces to 
\begin{equation}
\vec{M}_C =  \frac{d}{dt}\left(\sum\limits_{j=0}^P m_j{\bf S}(\vec{r}_{Cj}){\bf
  S}(\vec{r}_{Cj})^T\right)\vec{\omega}_{B/I} 
\end{equation}
Notice again that minus sign has been removed. The skew symmetric
matrix has an interesting property where the transpose is equal to the
negative of the original matrix. The term in brackets is a well known
value for rigid bodies and is known as the moment of inertia for rigid
bodies. 
\begin{equation}
{\bf I}_C =  \sum\limits_{j=0}^P m_j{\bf S}(\vec{r}_{Cj}){\bf
  S}(\vec{r}_{Cj})^T
\end{equation}
This results in the kinematic equations of motion for rigid bodies to
the simple equation below.
\begin{equation}
\vec{M}_C =  \frac{d}{dt}\left({\bf I}_C\vec{\omega}_{B/I} \right)
\end{equation}
With the equation in this form it is finally possible to carry out the
derivative
\begin{equation}
\vec{M}_C = \frac{{}^Bd ({\bf I}_C\vec{\omega}_{B/I})}{dt} + {\bf
  S}(\vec{\omega}_{B/I}){\bf I}_C\vec{\omega}_{B/I}
\end{equation}
The first term requires the chain rule to perform the derivative
however the body frame derivative of the moment of inertia matrix is
simply zero. Therefore the equation can simply be written as
\begin{equation}
\vec{M}_C = {\bf I}_C\frac{{}^Bd (\vec{\omega}_{B/I})}{dt} + {\bf
  S}(\vec{\omega}_{B/I}){\bf I}_C\vec{\omega}_{B/I}
\end{equation}
\section{Aircraft Convention}
Aircraft convention involves using the Newton-Euler equations of
motion to describe the aircraft\cite{etkins}. Typically the position of the
aircraft is written as 
\begin{equation}
{\bf C}_I(\vec{r}_C) = \begin{Bmatrix} x \\ y \\ z \end{Bmatrix}
\end{equation}
The derivative of the position vector is the velocity vector is then
written as
\begin{equation}
{\bf C}_I(\vec{v}_{C/I}) = \begin{Bmatrix} \dot{x} \\ \dot{y} \\ \dot{z} \end{Bmatrix}
\end{equation}
However, body frame coordinates are typically used to
describe the velocity vector such that
\begin{equation}
{\bf C}_B(\vec{v}_{C/I}) = \begin{Bmatrix} u \\ v \\ w \end{Bmatrix}
\end{equation}
In order to relate the body frame components of the velocity vector
the inertial frame coordinates a transformation matrix is used. The
transformation from the inertial frame to the body frame involves
three unique rotations. The first is a rotation about the z-axis such
that
\begin{equation}
{\bf C}_A(\vec{v}_{C/I}) = \begin{bmatrix} cos(\psi) & sin(\psi) & 0
  \\ -sin(\psi) & cos(\psi) & 0 \\ 0 & 0 & 1 \end{bmatrix} {\bf C}_I(\vec{v}_{C/I})
\end{equation}
this rotation is called the yaw or heading rotation. From here the
intermediate frame is rotated about the y-axis such that
\begin{equation}
{\bf C}_{NR}(\vec{v}_{C/I}) = \begin{bmatrix} cos(\theta) & 0 &
  -sin(\theta) \\ 0 & 1 & 0 \\ sin(\theta) & 0 & cos(\theta) \end{bmatrix} {\bf C}_A(\vec{v}_{C/I})
\end{equation}
this rotation is called the pitch angle rotation. Finally the no roll
frame is rotated through the x-axis such that
\begin{equation}
{\bf C}_{B}(\vec{v}_{C/I}) = \begin{bmatrix} 1 & 0 & 0 \\ 0 & cos(\phi) & sin(\phi)
  \\ 0 & -sin(\phi) & cos(\phi) \end{bmatrix} {\bf C}_{NR}(\vec{v}_{C/I})
\end{equation}
Putting all the equations together yields
\begin{equation}\label{e:xyzdot}
\begin{Bmatrix} \dot{x} \\ \dot{y} \\ \dot{z}   \end{Bmatrix} = [\textbf{T}_{IB}]
\begin{Bmatrix} u \\ v \\ w \end{Bmatrix}
\end{equation}
where 
\begin{equation}\label{e:TIB}
\textbf{T}_{IB} = \begin{bmatrix} c_{\theta}c_{\psi} &
s_{\phi}s_{\theta}c_{\psi}-c_{\phi}s_{\psi} &
c_{\phi}s_{\theta}c_{\psi} + s_{\phi}s_{\psi} \\ c_{\theta}s_{\psi} &
s_{\phi}s_{\theta}s_{\psi} + c_{\phi}c_{\psi} &
c_{\phi}s_{\theta}s_{\psi} - s_{\phi}c_{\psi} \\
-s_{\theta} & s_{\phi}c_{\theta} & c_{\phi}c_{\theta}
\end{bmatrix}
\end{equation}
Standard shorthand notation is used for trigonometric
functions: $ cos(\alpha) \equiv c_{\alpha} $ , $ sin(\alpha) \equiv
s_{\alpha} $ , and $ tan(\alpha) \equiv t_{\alpha} $. These three
angles are known as the standard Euler angle rotation sequence of an
aircraft in free flight. The angular
velocity of a body is typically written as 
\begin{equation}
{\bf C}_B(\vec{\omega}_{B/I}) = \begin{Bmatrix} p \\ q
  \\ r \end{Bmatrix} = p \hat{I}_B + q \hat{J}_B + r \hat{K}_B
\end{equation}
There are no inertial components for the angular velocity
vector. However, a relationship can be derived relating the
derivatives of the Euler angles. The angular velocity can be written
in vector form such that
\begin{equation}
\vec{\omega}_{B/I} = \dot{\psi} \hat{K}_A + \dot{\theta} \hat{J}_{NR} +
\dot{\phi} \hat{I}_B
\end{equation}
relating the unit vectors $\hat{K}_A$ and $\hat{J}_{NR}$ to the body
frame using the planar rotation matrices results in the equation
below. Note that NR is denoted as the ``No-Roll" frame.
\begin{equation}\label{e:ptpdot}
\begin{Bmatrix} \dot{\phi} \\ \dot{\theta} \\ \dot{\psi} \end{Bmatrix}
= [\textbf{H}]
\begin{Bmatrix} p\\q\\r\end{Bmatrix}
\end{equation}
where
\begin{equation}
\textbf{H}=\begin{bmatrix} 1 & s_{\phi}t_{\theta} & c_{\phi}t_{\theta} \\ 0 &
c_{\phi} & -s_{\phi} \\ 0 & s_{\phi}/c_{\theta} &
c_{\phi}/c_{\theta} \end{bmatrix}
\end{equation}
with all this information the Newton-Euler equations of motion can be
used to form the equation below.
\begin{equation}\label{e:uvwdot} 
\begin{Bmatrix} \dot{u} \\ \dot{v} \\ \dot{w} \end{Bmatrix} = 
\frac{1}m \begin{Bmatrix} X\\Y\\Z\end{Bmatrix}-\begin{bmatrix} 0 & -r
& q \\ r & 0 & -p \\ -q & p & 0 \end{bmatrix} \begin{Bmatrix} u\\v\\w\end{Bmatrix}
\end{equation}
Using the rotational dynamic equations for rigid bodies, the equation
for the derivative of angular velocity can be found as
\begin{equation}\label{e:pqrdot} 
\begin{Bmatrix} \dot{p} \\ \dot{q} \\ \dot{r} \end{Bmatrix} = {\bf I}_C^{-1}\left(
\begin{Bmatrix} L\\M\\N\end{Bmatrix}-\begin{bmatrix} 0 & -r
& q \\ r & 0 & -p \\ -q & p & 0 \end{bmatrix} {\bf I}_C\begin{Bmatrix} p\\q\\r\end{Bmatrix}\right)
\end{equation}
Note that standard aircraft forces and moments are applied to the
body. The forces are typically written as X,Y and Z while the moments
are given as L,M and N. They can be written in component form using
the equations below.
\begin{equation}
{\bf C}_B(\vec{F}_{C}) = \begin{Bmatrix} X \\ Y
  \\ Z \end{Bmatrix} = X \hat{I}_B + Y \hat{J}_B + Z \hat{K}_B
\end{equation}
\begin{equation}
{\bf C}_B(\vec{M}_{C}) = \begin{Bmatrix} L \\ M
  \\ N \end{Bmatrix} = L \hat{I}_B + M \hat{J}_B + N \hat{K}_B
\end{equation}

\section{Forces on Aircraft}
To form the sum of the forces on an aircraft the assumptions are made
that only gravity and aerodynamic forces act. For rockets, a
propulsion model can be added. Still the gravitational force is shown
below. 
\subsection{Gravity}
The weight contribution is assumed to be a constant force applied to
the aircraft. The equation below is the gravitational force applied in
the body frame.
\begin{equation}\label{e:wforce}
\begin{Bmatrix} X_W \\ Y_W \\ Z_W \end{Bmatrix} = mg \begin{Bmatrix}
-s_{\theta} \\ s_{\phi}c_{\theta} \\ c_{\phi}c_{\theta} \end{Bmatrix}
\end{equation}
\subsection{Aerodynamics}\label{s:aerodynamics}
Aircraft aerodynamics are written using a taylor series expansion
about a trim point\cite{Phillips,AndersonD}. That is, the aerodynamic forces are given by
\begin{equation}
\vec{F} = \vec{F}_0 + \frac{\partial \vec{F}}{\partial \vec{x}}(\vec{x}-\vec{x}_0)
\end{equation}
where $\vec{x} = [x,y,z,\phi,\theta,\psi,u,v,w,p,q,r]^T$. The partial
derivative is thus expanded such that
\begin{equation}
\frac{\partial \vec{F}}{\partial \vec{x}} = \begin{bmatrix} \frac{\partial
    \vec{F}}{\partial x} & \frac{\partial \vec{F}}{\partial y} & ... &
  \frac{\partial \vec{F}}{\partial r} \end{bmatrix}
\end{equation}
To find all of the partial derivative the forces are first written
using a combination of dynamic pressure and coefficients that are
functions of geometry and Reynolds number rather than speed, pressure
and size. A general lift force can be written using the equation below
\begin{equation}
L = \frac{1}2\rho {V_{\infty}}^2 S C_L
\end{equation}
where $\rho$ is the atmospheric density, $V_{\infty}$ is the
free-stream velocity, S is the planform area of the wing and $C_L$ is
the lift coefficient. 
\begin{equation}\label{e:vtotal}
  V_{\infty} = \sqrt{{u_a}^2 + {v_a}^2 + {w_a}^2}
\end{equation}
The subscript 'a' above denotes the velocity of the aircraft plus the
atmospheric disturbance. 
\begin{equation}\label{e:atm}
\begin{Bmatrix} u_a \\ v_a \\ w_a \end{Bmatrix} =
\begin{Bmatrix} u \\ v \\ w \end{Bmatrix} +
{\textbf{T}^T}_{IB} \begin{Bmatrix}
 V_x \\ V_y \\ V_z \end{Bmatrix}
\end{equation}
Note that the dynamic pressure is different for
a rocket or projectile. A similar expression can be created for a
generic moment such that
\begin{equation}
M = \frac{1}2\rho {V_{\infty}}^2 S \bar{c} C_M
\end{equation}
where $\bar{c}$ is the mean chord of the aircraft. The dynamic pressure $q_{\infty} =
\frac{1}2\rho {V_{\infty}}^2 S$ can be used to non-dimensionalize the forces, thus $L/q_{\infty} = C_L$. This means
that the equation involving partial derivatives can be written as
\begin{equation}
\frac{\partial \vec{C}_F}{\partial \vec{x}} = \begin{bmatrix} \frac{\partial
    \vec{C}_F}{\partial x} & \frac{\partial \vec{C}_F}{\partial y} & ... &
  \frac{\partial \vec{C}_F}{\partial r} \end{bmatrix}
\end{equation}
If the vector is then expanded to include the components of the vector
$\vec{F}$ the partial derivatives expand to
\begin{equation}
\frac{\partial \vec{C}_F}{\partial \vec{x}} = \begin{bmatrix}
  \frac{\partial C_X}{\partial x} & \frac{\partial C_X}{\partial y} &
  ... & \frac{\partial C_x}{\partial r} \\ \frac{\partial C_Y}{\partial x} & \frac{\partial C_Y}{\partial y} &
  ... & \frac{\partial C_Y}{\partial r} \\ \frac{\partial C_Z}{\partial x} & \frac{\partial C_Z}{\partial y} &
  ... & \frac{\partial C_Z}{\partial r} \end{bmatrix}
\end{equation}
shorthand can be adopted for the forces above such that
$\frac{\partial C_Y}{\partial x} = C_{Yx}$. Using this shorthand the
equation above can be written as.
\begin{equation}
\frac{\partial \vec{C}_F}{\partial \vec{x}} = \begin{bmatrix}
  C_{Xx} & C_{Xy} &
  ... & C_{Xr} \\ C_{Yx} & C_{Yy} &
  ... & C_{Yr} \\ C_{Zx} & C_{Zy} &
  ... & C_{Zr} \end{bmatrix}
\end{equation}
The coefficients listed above are standard coefficients that all
aircraft have. A similar matrix can be formulated for the moments on
an aircraft. When system identifying an aircraft all of these
coefficients may be determined. However, many of these terms are
zero. For example, all coefficients with respect to x y and z are
zero. That is, $C_{Xx} = C_{Yx} = ... C_{Nx} = C_{Xy} = ... C_{Nz} =
0$. Other coefficients can be set to zero as well. 
\subsubsection{Aircraft Aerodynamics}
For aircraft, some further simplifications are made. Some of the
coefficients defined above are combined to be written as functions of the
angle of attack($\alpha$) and sideslip($\beta$).
\begin{equation}\label{e:aoa}
\alpha = tan^{-1}\left(\frac{w_a} {u_a} \right)
\end{equation}
\begin{equation}\label{e:beta}
\beta = sin^{-1}\left(\frac{v_a}{V_{\infty}}\right)
\end{equation}
Transforming the equations into these formulations gives rise to
coefficients such as $C_{L\alpha}$ which is the change in lift as a
function of angle of attack and $C_{Y\beta}$ which is the change in
Y-Force as a function of sideslip. Using all of the coefficients
defined above taking into account the change to lift and drag, the body
aerodynamic force is calculated using the equation below.
\begin{equation}\label{e:aforce}
\begin{Bmatrix} X_A \\ Y_A \\ Z_A \end{Bmatrix} = \frac{1}2\rho
{V_{\infty}}^2 S \begin{Bmatrix}
C_Ls_{\alpha}-C_Dc_{\alpha}+C_{x_{\delta_t}} \delta_t
\\ C_{y{\beta}}{\beta}+C_{y{\delta}_r}{\delta}_r+C_{yp}\frac{pb}{2V_{\infty}}
+ C_{yr}\frac{rb}{2V_{\infty}} \\ -C_Lc_{\alpha}-C_Ds_{\alpha} \end{Bmatrix}
\end{equation}
Where the lift and drag coefficients are:
\begin{equation}\label{e:liftdrag}
\begin{Bmatrix} C_L \\ C_D \end{Bmatrix} = \begin{Bmatrix} C_{L0} +
C_{L\alpha}\alpha + C_{Lq}\frac{q\bar{c}}{2V_{\infty}} + C_{L\delta_e}\delta_e \\   C_{D0} +
C_{D\alpha}\alpha^2 \end{Bmatrix}
\end{equation}
The body aerodynamic moment is also computed using an aerodynamic expansion.
\begin{equation}\label{e:LMN}
\begin{Bmatrix} L_A \\ M_A \\ N_A \end{Bmatrix} = \frac{1}2\rho
{V_{\infty}}^2 S \bar{c} \begin{Bmatrix} C_{l\beta}\beta + C_{lp}\frac{pb}{2V_{\infty}} + C_{lr}\frac{rb}{2V_{\infty}} + C_{l\delta_a}{\delta_a} + C_{l\delta_r}{\delta_r}
\\  C_{m0} + C_{m\alpha}\alpha + C_{mq}\frac{q\bar{c}}{2V_{\infty}}+ C_{m\delta_e}\delta_e  \\ C_{np}\frac{pb}{2V_{\infty}} + C_{n\beta}\beta + C_{nr}\frac{rb}{2V_{\infty}} + C_{n\delta_a}\delta_a + C_{n\delta_r}\delta_r \end{Bmatrix}
\end{equation}
The aerodynamic coefficients in equations
(\ref{e:aforce}), (\ref{e:liftdrag}) and (\ref{e:LMN}) can be obtained from
flight data, aerodynamic modeling and windtunnel tests. Notice that
the only coefficients remaining are coefficients from angle of attack,
sideslip and angular velocities. Furthermore, the coefficients for
angular velocities are also non-dimensionalized by terms such as
$b/(2V_{\infty})$ where $b$ is the wingspan of the aircraft and
$\bar{c}$ is the mean chord of the aircraft. These terms are
introduced to fully non-dimensionalize the coefficients. Notice, as
well that four extra terms were also introduced. These will 
be discussed in more detail in the control section however the four
terms are the aileron control surface $\delta_a$, the elevator
control surface $\delta_e$, the rudder control surface $\delta_r$ and
the thrust control value $\delta_t$. 

\subsubsection{Projectile Aerodynamics}

To fully define the projectile aerodynamics some more assumptions are
made about the projectile.
\begin{enumerate}
\item The projectile is axially symmetric
\item The aerodynamic forces are not necessarily formulated at the center
  of mass
\item The projectile has the potential to be spinning rapidly thus
  interacting with the surrounding atmosphere
\end{enumerate}
For a projectile the dynamic pressure is written as
\begin{equation}
Q = \frac{\pi}{8}\rho V_{\infty}^2d^2 
\end{equation}
The aerodynamic forces on the projectile are modeled using taylor series
ballistic expansions with known coefficients similar to the aircraft
model only slightly different assumptions are made given the dynamics
of the projectile. The subscripts in the equation below stand for
steady and unsteady aerodynamics.
\begin{equation}\label{e:aeroF}
\begin{Bmatrix} X_A \\ Y_A\\ Z_A \end{Bmatrix} = \begin{Bmatrix}
  X_{SA} \\ Y_{SA}\\ Z_{SA} \end{Bmatrix} + \begin{Bmatrix}
  X_{UA} \\ Y_{UA}\\ Z_{UA} \end{Bmatrix} = 
Q \begin{Bmatrix} -C_{X_0} - C_{X_2}\frac{v^2+w^2}{V^2}\\
-C_{Y_\beta}\frac{v}{V}\\
-C_{N_\alpha}\frac{w}{V}
\end{Bmatrix} + Q \begin{Bmatrix} 0
  \\ C_{Y_{p\alpha}}\frac{w}{V}\frac{pd}{2V} \\ C_{Z_{p\alpha}}\frac{v}{V}\frac{pd}{2V}\end{Bmatrix}
\end{equation}
In this equation, $Q$ is the dynamic pressure, $d$ is the aerodynamic
reference area, $C_{X_0}$ is the zero-yaw axial force
coefficient,$C_{X_2}$ is the yaw-squared axial force coefficient,
$C_{N_\alpha}$ is the normal force derivative coefficient,
$C_{Y_{p\alpha}}$ is the Magnus force coefficient, and
$V=\sqrt{u^2+v^2+w^2}$ is the total velocity of the projectile. 
The aerodynamic moments acting on the projectile are the pitching,
pitch damping, Magnus, and roll damping moments. Pitching and Magnus
moments are given by taking the cross product of the normal and Magnus
forces given in (\ref{e:aeroF})  with the position vector from the
center of mass to the center of pressure and location of Magnus force,
respectively. The total aerodynamic moments are given in
Eqn. (\ref{e:aeroM}). 
\begin{equation}\label{e:aeroM}
\begin{Bmatrix} L_A \\ M_A \\ N_A \end{Bmatrix}=
{\bf S}_B(\vec{r}_{CG,COP})\begin{Bmatrix}
  X_{SA} \\ Y_{SA}\\ Z_{SA} \end{Bmatrix} + {\bf S}_B(\vec{r}_{CG,MCOP})\begin{Bmatrix}
  X_{UA} \\ Y_{UA}\\ Z_{UA} \end{Bmatrix} + 
Qd \begin{Bmatrix}
C_{l_p} \frac{pd}{2V} \\
C_{m_q} \frac{qd}{2V}\\
C_{n_r} \frac{rd}{2V} \end{Bmatrix}
\end{equation}
Here, ${\bf S}_B(\vec{r}_{CG,COP})$ is the skew-symmetric operator
acting on the position vector from the center of mass to the center of
pressure expressed in the projectile body frame. Furthermore, ${\bf S}_B(\vec{r}_{CG,MCOP})$ is the skew-symmetric operator
acting on the position vector from the center of mass to the magnus
center of pressure expressed in the projectile body frame. Typically
the center of mass is defined from the rear of the projectile such
that 
\begin{equation}
{\bf C}_B(\vec{r}_{CG}) = \begin{Bmatrix} SL_{CG} \\ BL_{CG} \\ WL_{CG} \end{Bmatrix}
\end{equation}
Similarly, the center of pressure is defined from the rear of the
projectile such that
\begin{equation}
{\bf C}_B(\vec{r}_{COP}) = \begin{Bmatrix} SL_{COP} \\ BL_{COP} \\ WL_{COP} \end{Bmatrix}
\end{equation}
The vector $\vec{r}_{CG,COP}$ is then simply the different between
both vectors.
\begin{equation}
\vec{r}_{CG,COP} = \vec{r}_{COP}-\vec{r}_{CG}
\end{equation}
The damping coefficient defined in equation (\ref{e:aeroM})
include $C_{l_p}$ which is the roll damping coefficient while $C_{m_q}$ is the pitch damping
coefficient. These coefficients are added which essentially inhibit
angular motion of the projectile. In addition, to these coefficients,
sometimes magnus coefficients are given as pure moments rather 
than forces acting at a distance. This can be given in the equation
below. 
\begin{equation}
M_{UA} = Qd (-C_{M\alpha}\frac{v}{V} + C_{N_{p\alpha}}\frac{w}{V}\frac{pd}{2V})
\end{equation}
Where $C_{M\alpha}$ replaces the moment produced by $C_{N\alpha}$ and
$C_{N_{p\alpha}}$ replaces the moment produced by
$C_{Y_{p\alpha}}$. It is possible to derive an equation between the
two different representations as given by the equations below.

\begin{equation}
\begin{matrix}
C_{M\alpha} = \frac{(SL_{COP}-SL_{CG})C_{N\alpha}}{d} \\
C_{N_{p\alpha}} = \frac{(SL_{MAG}-SL_{CG})C_{Y_{p\alpha}}}{d}
\end{matrix}
\end{equation}

\section{Stability and Control}

Controllability is formally stated as a system where any initial
state $x(0)=x_0$ and final state $x_1,t_1>0$, there exists a piecewise
continuous input $u(t)$ such that $x(t_1)=x_1$. 
For a fixed wing aircraft the system has 12 states with 8 dynamic
modes and 4 zero or rigid body modes. For a fixed wing aircraft the system has 12 states with 8 dynamic
modes and 4 zero or rigid body modes. A conventional aircraft has 4
controls to control these 12 modes. The easiest way to test the 
controllability of a system  is to compute the
controllability matrix. However, the controllability matrix must be
computed using a linearized model such that
$\dot{\vec{x}}=A\vec{x}+B\vec{u}$. In order to do this the aircraft
must be in equilibrium. For this example the aircraft is
set with an initial velocity of $20~m/s$ at an altitude of
$200~m$. The altitude command is set to $200~m$ and the heading
command is set to zero. Given the zero heading angle command and the
symmetry of the configurations investigated the rudder and aileron
commands are set to zero. Thus, only the thrust and elevator controls
are activated for the trimming procedure. Each configuration is
simulated for 200 seconds or until the derivatives of all states
except $\dot{x}$ are within a required tolerance. Using this
equilibrium point a linear model can be computed by using forward
finite differencing assuming that the
aircraft model is put in the form $\dot{\vec{x}} = F(\vec{x},\vec{u})$.
\begin{equation}
\dot{\vec{\delta x}} = \frac{F(\vec{x_0}+\Delta \vec{x_0},\vec{u_0})-F(\vec{x_0},\vec{u_0})}{\Delta
  \vec{x}}\vec{\delta x} + \frac{F(\vec{x_0},\vec{u_0}+\Delta
  \vec{u})-F(\vec{x_0},\vec{u_0})}{\Delta \vec{u}}\vec{\delta u}
\end{equation}
This linear model is the classic linear model where
$\dot{\vec{\delta x}}=A\vec{\delta{x}}+B\vec{\delta{u}}$. Using this linear model, the
controllability matrix can be computed as
\begin{equation}
W_C = [B~AB~A^2B~A^3B~...~A^{N-1}B]
\end{equation}
where N is the number of states in the system. With the controllability
matrix formulated, the rank of the matrix is computed. If the
$rank(W_C)=N$ the system is said to be controllable.

\section{PID Control}

For a conventional PID controller of an aircraft, the rudder, elevator
and aileron commands are set to 
\begin{equation}
\begin{matrix}
\delta_{r} =-K_v v \\
\delta_{e} = K_p(\theta-\theta_{c})+K_d{\dot \theta} \\ 
\delta_a = K_p(\phi-\phi_{c})+K_d{\dot \phi}
\end{matrix}
\end{equation}
The Euler angle commands $\phi_{c}$ and $\theta_{c}$ are set using the
following relationships:
\begin{equation}
  \begin{matrix}
  \phi_{c} = K_p(\psi-\psi_c)+K_d\dot{\psi} \\
  \theta_{c} = K_p(z-z_c) + K_d\dot{z} + K_{I}\int{z-z_c}dt
  \end{matrix}
\end{equation}
The control scheme defined above is a conventional inner loop-outer
loop control of a fixed wing aircraft using a PID tracking
controller.

\section{Radio Controlled Aircraft Design}

Many principles for large aircraft design can be applied to smaller
radio controlled aircraft but understand that many of the aerodynamic
principles are not quite well defined for slow and small
aircraft. These aircraft have low Reynolds number which often exhibits
odd phenomena. For example, airfoil selection is really not so much a
design point other than ease of manufacturing rather than maximizing
lift coefficient. Remember that Reynolds number can be defined from
the equation below where $\rho$ is the density at sea-level (1.225
kg/$m^3$ in SI units), $V$ is
the desired flight speed, $\bar{c}$ is the mean aerodynamic chord and
$\mu_{\infty}$ is the viscosity of air which in SI units is 1.81e-5
$kg/(m-s)$. 

\beq
Re = \frac{\rho V \bar{c}}{\mu_{\infty}}
\eeq

You can tell that flying a small aircraft ($\bar{c}$) and flying slow
$(V)$ results in a low Reynolds number. Either way the procedure below
has produced some great aircraft and the tools you'll learn along the
way will help you in your future aerospace engineering career. If
you're not an engineer then this text will at least give you an
appreciation for what goes into aircraft design. If you'd much rather
watch youtube videos than read this document, feel free to watch my
Youtube playlist on radio controlled aircraft design\cite{RCYoutube}

\subsection{Vehicle Type Selection and Requirements}

In the very beginning of your design you need to decide on the type of
aircraft you want to build. Designing a glider versus an aerobatic
airplane will result in vastly different engineering design
decisions. For example, a glider is going to have very long slender
wings while an aerobatic airplane is going to have somewhat shorter
wings with large control surfaces. I suggest you select from the
following types of aircraft and then move on: Gliders, Trainers, Sport
Aerobatic, Racers. If you'd like to build a scale aircraft there isn't
much to design since the shape of the aircraft is pretty much
built. If you do go with a scale aircraft this textbook isn't really
for you since you aren't really building a scratch build
aircraft. You're more just copying someone else's design. In that case
you may as well just buy a kit or watch some videos on balsa wood
construction and an overview of all the electronics required for RC
aircraft flight.

If you selected one of the other styles you're ready to move onto to
the next stage which is requirements. There is so much literature on
Systems Engineering, top level requirements, functional
requirements and derived requirements. The bottom line is you need to
determine what you want your aircraft to do. Do you want it to fly
straight up, upside down? Do you want the aircraft to be hand
launched? Land on a runway? Determine what you want the aircraft to do
and create a bulleted list of those requirements. Throughout the
design you can refer to these requirements and make sure you are
satisfying these requirements. If this is your first build then you
may just have one requirement and that is to take off and land without
crashing. But think a bit deeper. Do you want to turn the vehicle? Do
you want full channel control for roll, pitch and yaw or just yaw
control? Do you want landing gear? What sort of flying characteristics
do you want? Be as specific as possible here. 

\subsection{Initial Design - Hand Sketch and Aspect Ratio}

Once you have an idea of what the aircraft type is and what the
requirements are it's time to hand sketch your aircraft. Try and use
engineering paper, french curves, a ruler and a compass. Make this
hand sketch look nice so you can use it in your future design. Draw
your sketch to scale. You might be wondering, ``how do I draw the
aircraft without knowing what my wing loading or thrust to weight
ratio is?". The answer comes from my late aircraft design professor
Dr. Mikolowski (RIP). He would always say ``If it looks good, it flies
good". After designing so many aircraft and seeing so many scratch
builds from my students I can honestly say that this is true
100\%. If you're reading this now it means that there is already over
100 years of aircraft technology on the internet for you to research
and see what other aircraft look like. Make your aircraft look like
that but make sure it fits into your aircraft type and make sure it
satisfies your requirements from above. If one of your requirements
was to hand launch, then make sure your drawing reflects a vehicle
without landing gear and a place to grab the aircraft. If you wanted
full channel support make sure to include all the control
surfaces. Think about where you want the propulsion system to go and
how you're going to access the electronics before you fly. Think about
what you want the wing to look like. Make it as big as you think it
needs to fly. Use your intuition. This is an art. So much of
engineering is an art.

Once your aircraft sketch is complete (make sure to do a front view,
side view and top view of your aircraft), it's time to take down some
wing characteristics. This is why you need to draw your sketch to
scale. Measure the length of the wing (wingspan $b$) and the chord at the
root ($c_r$) and the tip ($c_t$). Compute the area of the wing ($S$) using
the area of a trapezoid or rectangle depending on the shape of your
wing. Once you have the wingspan and area you can compute the aspect
ratio of your aircraft.

\beq
AR = \frac{b^2}{S}
\eeq

The general rule is that the larger the aspect ratio the more
aerodynamically efficient your aircraft will be. This is why gliders
have very long and slender wings. At the same time, high aspect ratio
wings suffer from larger bending moments and can flex considerably in
flight. Finally, it's important to compute the mean aerodynamic chord
of the wing. This is basically the average chord of your wing. If you
create a rectangular wing your mean aerodynamic chord is just the
chord of the wing since it's constant. If not you'll need to integrate
over the length of the wing using the formula below where y = 0 is the
centerline of the vehicle and y=b/2 is the wingtip on the right
side\cite{caughey}. The parameter $c(y)$ is the chord length as a
function of y.

\beq
\bar{c} = \frac{2}{S}\int^{b/2}_0 c(y)^2 dy
\eeq

\subsection{Weight Estimate - Tabular Approach}

Once you have an idea of the overall shape it's time to estimate how
heavy the aircraft will be. First think about the fundamental
components of an aircraft. If you're not familiar with any of the
components below, just type the item into Google and you'll find
numerous articles and Youtube Videos about each component.

\begin{enumerate}[itemsep=-5pt]
\item ESC - Electronic Speed Controller
\item Battery - Assume for now that you'll be using a 1500mAh 3S 
  Battery unless you're building a micro aircraft in which case you
  might end up using a 600 mAh 2S or even a 300 mAh 1S. Think about
  the size of your aircraft. You will do more sophisticated battery
  design in the future.
\item Motor and Propeller - Again select something that is in the
  ballpark of the aircraft you're building. You'll do a redesign later.
\item Servos
\item Receiver
\item Control Linkages and Servo Horns
\item Fuselage
\item Main Wing
\item Tail both Horizontal and Vertical
\item Payload
\end{enumerate}

For each of the components above you need to estimate the weight of
these components. The only way to do that is to either look up the
weight of similar aircraft to the one you're designing or find
components that you think will work for your aircraft and add up all
of the weights. The most difficult part is going to be estimating the
empty weight of the aircraft which is just the structure of the
aircraft. For these estimates you need to decide what materials you
plan on using. Is your aircraft made out of foam, balsa, carbon fiber
or some type of combination. Perform an initial material selection and
then use that to estimate your weight. Create a table in a spreadsheet
type program or a numerical computer program so that you can go back
and change weights as your design progresses. Use the spreadsheet or
numerical program to compute the total weight of your aircraft. This
is your maximum takeoff weight.

\subsection{Airfoil Selection and 2D and 3D Lift}

The section \ref{s:aerodynamics} details the aerodynamic forces on the
aircraft. In this stage of design we will only be looking at the lift
equation.
\begin{equation}
L = \frac{1}2\rho {V}^2 S C_L
\end{equation}
The coefficient of lift $C_L$ is the lift coefficient of the
aircraft. This coefficient is a function of Reynolds number, angle of
attack, airfoil shape and wing shape. Recall that the angle of attack
is the angle between the zero lift line of the airfoil and the free
stream air. Breaking the velocity vector into components yields the
equation below where $w$ is the airspeed along the z-axis of the
vehicle and $u$ is the velocity along the x-axis. 
\begin{equation}\label{e:aoa}
\alpha = tan^{-1}\left(\frac{w} {u} \right)
\end{equation}
Using the angle of attack as a parameter, the lift coefficient can be
expanded to the following:
\beq
C_L = C_{L\alpha}(\alpha-\alpha_0)
\eeq
The parameter $\alpha_0$ is the angle of attack that results in zero
lift. This is a function of the airfoil shape. The parameter
$C_{L\alpha}$ is the lift curve slope and is a function of airfoil
shape, wing shape and Reynolds number. In order to remove, wing shape
from the design, the aspect ratio is used to convert the wing lift to
a sectional airfoil coefficient.
\beq\label{e:ar_equation}
C_{L\alpha} = \frac{C_{l\alpha}}{1+\frac{C_{l\alpha}}{\pi e AR}}
\eeq
The parameter $e$ is an efficiency parameter which is often assumed to
be 80-90\%. The coefficient, $C_{l\alpha}$ is the airfoil lift curve
slope which is different than the wing lift curve slope. The wing lift curve slope
will always be smaller than the airfoil lift curve slope. This is
because of an effect called wing tip vortices. Wing tip vortices are
an aerodynamic effect where high pressure from the bottom of the wing
moves around the wingtips to the area of low pressure. The only way to
mitigate these vortices is by installing winglets or increasing the
aspect ratio. Note that winglets increase drag and weight which is why
you don't typically see them on RC aircraft. The coefficient
$C_{l\alpha}$ is then simply a function of airfoil shape and
Reynolds number. This is what is often referred to as 2D lift. It is
the lift of airfoils which are two dimensional rather than 3D lift
which is over an entire wing. It is at this point that airfoil
selection and design can be used. The website \url{airfoiltools.com}
is a great resource for plotting lift curve slopes of various airfoil
shapes as a function of Reynolds number. I also have a great Youtube
video on how to use XFLR5 (pronounced X-Flyer
Five)\cite{XFLR5_tutorial}. A more comprehensive guide on XFLR5 can be
found in \cite{XFLR5_Guidelines} while the software itself can be
downloaded in \cite{XFLR5}. The basics of airfoil selection then break
down into the following 
process.

First, using your max takeoff weight and cruise flight speed, compute
the lift coefficient required in cruise. This assumes that lift equals
weight. 
\beq
C_L = \frac{2W}{\rho V^2 S}
\eeq
Then using your flight speed and chord length, compute the Reynolds number
of your aircraft. Use this Reynolds number and airfoiltools or XFLR5
to compute the sectional lift characteristics of various
airfoils. Airfoil selection can be as complicated as you make it but
you're looking for the highest lift to drag ratio airfoil. Another
simple way is to just select the airfoil with the highest sectional
lift coefficient. Another item to consider is manufacturing. Some
airfoils may be feasible for full sized aircraft but not for RC
flyers. My recommendation is to build an aircraft with a Clark-Y
airfoil. They are easy to cut with balsa or shape with foam and have
good lift to drag characteristics. Note that this airfoil is
cambered. If you want to fly upside down I suggest you use a symmetric
airfoil like a NACA 0012 or NACA 0014 is you need a bit more thickness
to fit a wing spar through the airfoil.
Once you have your airfoil selected, use the lift curve slope to
estimate $C_{l\alpha}$ by fitting a linear trend line to the portion
of the graph before stall. Also make sure to take note of what the
zero lift angle of attack is. You can then compute the wing lift curve
slope by using equation \ref{e:ar_equation}. Once you have
$C_{L\alpha}$, you can compute the angle of attack needed for
cruise. Make sure to convert $\alpha_0$ to radians before you use the
equation below.
\beq
\alpha = \frac{C_L}{C_{L\alpha}}+\alpha_0
\eeq
The resulting answer will be in radians and needs to be converted to
degrees to make sure you are not close to stall. Typically in cruise,
the angle of attack is only a few degrees with perhaps 8 degrees on
the high end. If the answer you receive is higher than that it can
mean a few things which may involve a redesign.
\begin{enumerate}[itemsep=-5pt]
  \item The simplest way to get more lift is to fly faster. The
    problem is you need a bigger motor which will increase your weight
    which will require more lift and more angle of attack. Flying
    faster will also change your Reynolds number. 
  \item You can increase the aspect ratio of your wing which will make
    your aircraft more efficient. The problem is that will increase
    the bending moment at the root creating the need for stronger
    materials at the root which also increases weight. This will also
    change your mean aerodynamic chord which will change your Reynolds
    number. 
  \item You can increase the area of the wing. This will also increase
    drag and weight but if the material you are using has a high lift
    to weight ratio then adding more wing area might be a good
    option. Depending on how you change the aircraft wing shape, the
    aspect ratio and/or the mean aerodynamic chord might change
    meaning you'll have to recompute the wing lift curve slope as well
    as the Reynolds number. 
  \item If you are using a symmetric airfoil it's possible you could
    forgo flying upside down and switch to a cambered airfoil which
    has more lift.
\end{enumerate}

Regardless of what you do make sure you angle of attack in cruise is
low which will reduce drag. It's optimal to fly at the aircraft's
highest lift to drag ratio but radio controlled aircraft don't
typically do that. It is also important to compute the stall speed of
your aircraft. You'd like this value to be as small as possible.
\beq
V_{stall} = \sqrt{\frac{2W}{\rho S C_{Lmax}}}
\eeq
In this case $C_{Lmax}$ is the maximum lift coefficient your aircraft
can obtain before stalling. If the stall speed is too high for your
design go back and redesign your vehicle. Once you have redesigned
your vehicle to fit within tolerable limits it's time to look at some
aircraft performance characteristics which include the W/S (wing
loading) and the T/W (thrust to weight ratio). 

\subsection{Wing Loading and Thrust to Weight Ratio}

The wing loading is defined as the weight of the aircraft over the
main wing area (W/S). Intuitively though, it is the amount of lift
required per square foot of wing area for your aircraft. If the wing
loading is high it means you have a heavy aircraft with small wings
which means you either need very high lift creating devices like flaps
and cambered airfoils or the aircraft needs to fly very fast. You can
imagine that warbirds and racers have higher wing loading then say a
glider which has very low wing loading. In this case the aircraft can
fly slow because the aircraft is light with larger wings. At this
stage of the design you already know the maximum weight of the
aircraft and the main wing area so it's simple to calculate. For
larger aircraft, there is a standard wing loading for aircraft as well
as another parameter called the wetted wing loading which is the
weight divided by the wetted area of the wing. The wetted area is
basically the surface area of the wing. For radio controlled aircraft
though the wing cube loading is used to ensure the aircraft is of the
correct type.
\beq
WCL = \frac{W}{S^{3/2}}
\eeq
Using the units of ounces for the weight and sqft for the area the
table below can be used to ensure that your aircraft is in the correct
ballpark\cite{WCL}.
\begin{table}[H]
  \begin{center}
  \begin{tabular}{cc}
    Type of Aircraft & WCL ($oz/ft^3$) \\
    \hline
    \hline
    Gliders & under 4\\
    \hline
    Trainers & 5-7\\
    \hline
    Sport Aerobatic & 8-10\\
    \hline
    Racers & 11-13 \\
    \hline
    Scale & over 15 \\
  \end{tabular}
  \end{center}
\end{table}
After computing your WCL it is possible that for the type of aircraft
you've designed, your value falls well outside the limits of the table
above. In that case you must redesign your vehicle by changing the shape
of the wings or choosing a different material to lighten up the
aircraft. In my experience, if you are designing a racer or scale
aircraft and you wing loading is smaller than above this is typically
ok so long as you have enough thrust to fly fast. In this case you may
just have a very fast trainer which isn't necessarily a bad thing. The
danger is when trying to design a glider with a WCL of over 15. That
aircraft will exhibit a very high stall speed and poor lift to drag
ratio (L/D) characteristics. Once you are satisfied with your WCL you
can move on to computing the thrust to weight ration (T/W).

For full-sized aircraft the T/W can range from 0.6 for passenger
aircraft all the way up to 1.2 or higher for get aircraft. For R/C
aircraft the same general rule applies. If your aircraft is a trainer
with landing gear you can probably get away with a T/W of 0.6 but I
would not recommend it. My recommendation would be to go no lower than
0.8 which would mean your maximum take off thrust is 80\% of your
maximum takeoff weight. In this configuration, the aircraft will
accelerate down the runway until just over stall speed at which point
the aircraft can takeoff. If you have strict runway requirements,
airborne requirements like vertical flight or loops and snap rolls, I
recommend increasing the size of your motor, ESC, battery, propeller
combination to yield a T/W of at least 1.2. In this case, even if your
wings are not very efficient, you can fly on thrust alone. You may not
exhibit great aerodynamic performance and still have a high stall
speed but worst case you can land the aircraft like a harrier which
I've done before.

Once you've selected your T/W you need to go find a
battery/ESC/motor/propeller combination that yields the thrust you
need. Tiger Motors website is typically very good at listing the
motor, propeller and battery combination to give you a certain amount
of thrust. Unfortunately, the hobbyist market is not using standard
engineering units and thrust is reported in grams. My recommendation
then is to use the following formula to compute your thrust required
in grams. This assumes that 4.44 N = 1 lbf and that you are on Earth
with 9.81~$m/s^2$ of gravitational acceleration. It'd be nice if the
community just reported thrust in lbf but alas that is not the case. 

\beq
T_{grams} = (T/W)W_{lbf}/453.59
\eeq

Remember, when selecting a propulsion system, you will need to go back
and update your weight estimate with the new values you've
obtained. This may effect your wing area slightly and may even require
you to choose a new motor if you were very off the first time you
estimated the weight. This is an iterative process and every step
builds on the previous step. The hope is that each iteration is not
very different than the last. 

\subsection{Stability and Control, Center of Mass, Aerodynamic Center and Static Margin}

Stability and Control is a very large section of literature and could
be taught over an entire semester. Stability just ensure the aircraft
flies steady and level and is typically broken up into lateral (side
to side) and longitdutinal (front to back) stability. Lateral
stability in my opinion is more complex but to ensure you aircraft is
laterally stable, just be sure your aircraft is symmetric about the
left and right planes and also be sure that your tail surface provides
adequate yaw stability through the use of a vertical tail or a V-tail
if you opted for a combined control surface. Longitudinal stability
involved two more calculations that must be done before the aircraft
can be built. These two parameters are the center of mass and the
aerodynamic center. The center of mass is a very simple quantity to
compute by just using the center of mass formula shown below.
\beq
x_{cm} = \sum\frac{x_iW_i}{W}
\eeq
In the equation above, $x_i$ is the distance of a component from a
reference point on the aircraft. I typically use the nose of the
fuselage as the reference point. For standard aircraft the motor would
have a negative distance from the reference point and the receiver and
battery would have a positive value. The value $W_i$ is then the
weight of each component. Placing servos, receivers and other
electronics is a design parameter to move your center of mass while
the fuselage is typically a parameter that must be estimated at this
stage. My recommendation is to break the aircraft into fuselage, tail
boom, tail and main wing components and treat each one as a
component. You will notice that moving the main wing and battery
drastically changes the center of mass.

The next parameter is the aerodynamic center. The main wing looking
from the top is basically a 2D distributed load. As such the center of
lift must be computed. Assuming the main wing is symmetric, the
aerodynamic center will lie on the center line of the aircraft. In
this case the problem reduces to a 1D computation. In order to compute
the center of lift along the x-axis (pointing towards the nose) you
need to compute the weighted average of the center of lift of each
airfoil. In this case if you have a symmetric airfoil, the center of
lift is 1/4 of the chord length. If you have a cambered airfoil the
center of lift is typically in the 30\% range so you can use 25\% for
a symmetric airfoil and something slightly larger for a cambered
airfoil. Once you determined the center of lift for the airfoil you
can use the equation below for the aerodynamic center of the entire
wing\cite{AndersonD,caughey}.
\beq
x_{ac} = \frac{2}{S}\int^{b/2}_0 x_{af}(y)c(y) dy
\eeq
The parameter $x_{af}(y)$ is the location of the center of lift of the
airfoil as a function of y. Once you have the aerodynamic center and the
center of mass you can compute the static margin of your aircraft.
\beq
S_{m} = \frac{x_{ac}-x_{cm}}{\bar{c}}
\eeq
The value of the static margin is not as important as the sign. If
measuring from the nose of the aircraft the aerodynamic center must be
behind the center of mass. To explain this think about stable
aerodynamic vehicles like darts, or arrows. Notice that darts and
arrows have fletching in the rear to create aerodynamic surfaces
farther back. Unstable vehicles like frisbees and footballs have
aerodynamic centers in front of the center of mass in which case they
must spin in order to provide stability just like a bike tire or a
dreidel. Using the equation above, if $x_{ac}$ is behind $x_{cm}$ it
means that $x_{ac}$ is bigger or more positive than $x_{cm}$ in which
case your static margin would be positive.
\beq
\begin{Bmatrix} S_m>0,~stable \\ S_m<0,~unstable \end{Bmatrix}
\eeq
If you perform these two calculations and find your static margin to
be negative it means that you need to shift your battery and other
components more towards the nose or move your wings backwards. Note
that moving your wing backward will also shift your center of mass so
try and move some components forward before shifting your wings
around.

The final stage of this design is Control. Aircraft in flight require
3 control surfaces to provide roll, pitch and yaw control. These
include the aileron, elevator and rudder. It's possible to fly
aircraft without a rudder by performing a ``bank and yank" maneuever
and it's also possible to combine elevators and ailerons into
something called elevons. I've even seen some ruddervators. Whatever
you decide to do make sure that you can adequately control all three
axes or if one axis is uncontrollable be sure that that axis is
stable. I've seen some aircraft that only have rudder and elevator and
no ailerons. I find these aircraft hard to control but the idea is you
move the rudder to yaw the aircraft which also rolls the aircraft
allowing you to turn. It creates a very slow aircraft but it also
reduces complexity if that is something you're interested in doing. 

\subsection{Iteration, Detailed Sketch and Final Checks}

This section in my opinion is absolutely essential. It involves going
back and making sure that your current design satisfies your
requirements you originally wrote in the first section of this
design. Recompute your aspect ratio, and update your weight estimate
based on any calculations you've obtained. This may be finding better
estimates for parts or materials. You also need to create a better
sketch and determine where EVERY component is going to go and what
sort of support you will need. If you're building your aircraft out of
balsa you will need detailed sketches on rib, spar and stringer
placement. All of these updates will change your weight estimate which
will change your WCL and your T/W. Be sure your WCL and T/W are within
tolerable bounds. You also need to go back and compute you angle of
attack during cruise and be sure you are not in a stall regime. Be
realistic with your flight speed as well during cruise. Go back and
compute your stall speed. Is it realistic? If not then go back and
make some minor changes. Finally, be sure you aircraft is
longitudinally and laterally stable and that you can control all 3
axes or at least the uncontrollable axes are stable. Once you are
certain the aircraft will fly you can begin purchasing components. 

\subsection{Computer Aided Design (CAD)}

This section is optional but sometimes it's just nice to have a CAD
view of your aircraft especially if you are 3D printing parts or
perhaps getting some component machined out of aluminum. Some CAD
programs are even so powerful they will compute bending loads, center
of mass and even drag. Use whatever tools are at your disposal to help
you in the design. 

\subsection{Purchase Components}

I wanted to write an entire section on purchasing component to go over
a few common mishaps. First, if you are using a LiPo battery be sure
to familiarize yourself with the dangers of LiPo batteries. I've
almost caught my entire lab on fire by charging a damaged LiPo but all
batteries can technically catch fire. Please be careful. Furthermore,
be sure you purchase an ESC with the right current rating. If you
overload an ESC with too much current it will also catch on fire. The
servos you buy have a torque rating. Be sure your servos can overcome
the aerodynamic torque estimated in flight. Generally servos are sized
by the size of the aircraft so you can just purchase servos that are
designed for your particular size aircraft. The motor you purchase is
going to need a mounting point. Consider designing a motor plate or
even a firewall depending on the type of aircraft. When purchasing
materials make sure to get them from a good brand. Companies like
Flite Test sell really good double plated foam that is designed for RC
aircraft. Purchasing Dollar Tree foam board is totally acceptable but
just understand that after a crash or two the foam board won't work
anymore. Also consider what type of glue you are planning on
using. Certain types of glue can actually melt foam and hot glue can
melt certain types of foam as well. CA glue is very good for balsa but
will melt foam. Two part epoxy is strong but it weighs more than CA
glue and takes a long time to set compared to other types of
glue. Also be sure to be lenient on glue where you can. Glue just adds
weight and that will reduce your performance. Finally, make sure your
receiver supports the number of servos you plan on using and be sure
that your transmitter and receiver are compatible. All of my aircraft
use Spektrum technology but you may opt to use a different type of
protocol.

Finally, when all components come in be sure to test them. DOA stands
for dead on arrival and so many components come DOA. Before you spend
the time to install all your components in your aircraft and then
throw the aircraft in the air make sure you test every component and
be sure it works. I also suggest weighing each component on a small
scale and updating your weight estimate to ensure everything is within
tolerable bounds. 

\subsection{Building}

Once you have all necessary resources to build your aircraft I
recommend starting with the fuselage or main wing and then installing
all componets. I recommend taking pictures of your build in case you
need to reference them later. Go slow. If you break something it will
be expensive. Also remember that if the aircraft looks good it flies
good. This means that gluing all components properly and having the
aircraft be as smooth as possible will translate to better flight
performance. 

\subsection{Flying}

Before you fly I recommend finding a pilot will more experience than
yourself to check out the aircraft and make sure the aircraft has been
built properly. You may even want to discuss your initial design
before you even begin building to make sure there are no major
critical issues. If you want to fly the aircraft yourself I suggest
flying an aircraft using a simulator. My recommendation is the free
program CRRCSim\cite{CRRCSim}. It is not a great simulator by any means but it will
at least familiarize yourself with aircraft controls. Before you fly
make sure to build yourself a pre and post flight check list. I've
included my pre and post flight check list that I use before every
flight test.

\subsubsection{Day Before Flight Checklist}
\begin{enumerate}[itemsep=-5pt]
\item Assess the weather to ensure acceptable flight conditions
  \begin{enumerate}[itemsep=-5pt]
    \item No strong winds (insert windspeed conditions)
    \item No rain or lighting
  \end{enumerate}
\item State and confirm the purpose of the flight test - Set clear
  goals the aircraft should complete before test 
\item Check for damage to the plane and if the moving parts are secured including motor and electronic speed control and all components
\item Check for a full battery charge on plane and controller; charge
  all electronics if not fully charged
\item Perform Ground Safety Check List
\item Take note of items that need to be repaired even if the flight test is not implemented
\end{enumerate}

\subsubsection{Ground Safety Check List}
\begin{enumerate}[itemsep=-5pt]
  \item Ensure that propellor is off
  \item Turn on TX
  \item Connect battery to aircraft
  \item Ensure all control surfaces are operational and moving the
    correct way
  \item Spin up motor and be sure that motor is spinning the correct
    way
  \item Perform a range check where pilot moves control surfaces with
    50\% or more throttle while walking away from aircraft. Ensure that pilot can move at least
    300 feet away without any dropouts.
  \item Remove battery and install propeller
  \item Reconnect battery.
  \item Using safety glasses, apply full throttle to TX and ensure
    that adequate thrust is generated to fly aircraft. Leave full
    throttle applied for at least 30 seconds to be sure no component. 
    fails. Better to fail on the ground than in the air.
\end{enumerate}

\subsubsection{Preflight Checklist}

\begin{enumerate}[itemsep=-5pt]
\item Perform all “Day Before Flight” Checks
\item Perform Ground Safety Checks
\item Check for any damage to any components including the battery
\item Install Prop
\item Turn on TX
\item Plug in main battery
\item Confirm flight time and range distance
\item Clear obstructions and make sure there is clear space for takeoff and landing
\item Arm TX if the TX has an arm switch
\item Ensure all control surfaces are operational and moving the correct way
\item Apply throttle and fly
\item Upon landing do everything in reverse order
\end{enumerate}

\subsubsection{Post Flight checklist}
\begin{enumerate}[itemsep=-5pt]
  \item Check plane for any damage
  \item Check all moving parts are still secured
  \item Check for battery overheating, discoloration, warping, or swelling
  \item Check battery usage with a voltmeter
  \item Check if the plane is able to power on again
  \item Have PIC (pilot in command) give a post flight assessment
  \item Put batteries in LiPo storage
\end{enumerate}

\bibliographystyle{unsrt}
\bibliography{../papers.bib}

\newpage

\section{AIRCRAFT NOMENCLATURE}
\begin{tabbing}
  XXXXXXXXXX \= \kill% this line sets tab stop
  $x_i,y_i,z_i$ \> components of the mass center position vector in the
  inertial frame of aircraft $i$ (m) \\
  $\phi_i,\theta_i,\psi_i$ \> Euler roll,pitch, and yaw of aircraft
  $i$ (rad) \\
  $u_i,v_i,w_i$ \> components of the mass center velocity vector in the
  body frame of aircraft $i$ (m/s) \\
  $p_i,q_i,r_i$ \> components of the mass center angular velocity vector in the
  body frame of aircraft $i$ (rad/s) \\
  ${\vec r}_{A\rightarrow B}$ \> position vector from a generic point A
  to a generic point B(m) \\
  ${\vec V}_{A/B}$ \> velocity vector of a generic point A with respect
  to frame B (m/s) \\
  $\textbf{T}_{AB}$ \> generic transformation matrix rotating a vector from
  the frame B to frame A \\
  $\textbf{H}_i$ \> relationship matrix of Euler angle derivatives to
  body angular velocity components of aircraft $i$ \\
  $m_i$ \> mass of aircraft $i$ (kg) \\
  $I_i$ \> moment of inertia matrix of aircraft $i$ taken about the mass center
  in the body frame($kg-m^2$) \\
  $X_i,Y_i,Z_i$ \> components of the total force applied to aircraft $i$ in
  body frame(N) \\
  $L_i,M_i,N_i$ \> components of the total moment applied to aircraft
  $i$ in body frame(N-m) \\
  $X_{Wi},Y_{Wi},Z_{Wi}$ \> total weight force applied to aircraft
  $i$ (N) \\
  $L,D$ \> Lift and Drag on Aircraft (N) - Not to be confused with Roll moment\\
  $g$ \> gravitational constant on Earth $(m/s^2)$ \\
  $\rho$ \> atmospheric density($kg/m^3$) \\
  $S_i$ \> reference area of wing on aircraft $i$ ($m^2$) \\
  $b_i$ \> Wingspan of aircraft $i$ (m) \\
  $\bar{c}_i$ \> mean chord of wing on aircraft $i$ (m) \\
  $\alpha$ \> Angle of attack (rad) \\
  $\beta$ \> Slideslip angle (rad) \\
  $C_L,C_D,C_m$ \> Lift, Drag and Pitch Moment coefficients \\
  $\delta_t,\delta_a,\delta_r,\delta_e$ \> thrust, aileron, rudder,
  and elevator control inputs(rad) \\
  $S_B(\vec{r})$ \> skew symmetric matrix operator on a vector
  expressed in the body frame. \\
  $K_p,K_d,K_I$ \> proportional, derivative, and integral control
  gains\\
  $V$ \> Total airspeed (m/s) \\
  $\hat{p},\hat{q},\hat{r}$ \> Non-dimensional angular velocities \\
  $l$ \> Distance from center of mass to aerodynamic center of the
  tail (m) \\
  $l_t$ \> Distance from aerodynamic center of main wing to
  aerodynamic center of tail (m) \\
  $\alpha_0$ \> zero lift angle of attack (rad) \\
  $C_{L0}$ \> Zero angle of attack lift coefficient \\
  $C_{m\alpha}$ \> Pitch moment curve slope versus $\alpha$ \\
  $C_{L\alpha}$ \> Lift curve slope \\
  $C_{mq}$ \> Pitch damping coefficient \\
  $C_{m\delta_e}$ \> Pitch moment curve slope versus elevator
  deflection angle \\
  $a_{\infty}$ \> Speed of sound (m/s) \\
  $\mu_{\infty}$ \> Viscosity of Fluid $kg/(m-s)$ \\
  
\end{tabbing}

\newpage

\section{EQUATIONS}

\begin{multicols}{2}

\noindent Mach Number and Reynolds Number

\beq
\beqn
M_{\infty} = \frac{V}{a_{\infty}} \\
\ \\
Re = \frac{\rho V \bar{c}}{\mu_{\infty}}
\eeqn
\eeq

\noindent Total Velocity

\beq
V = \sqrt{u^2 + v^2 + w^2}
\eeq

\noindent Angle of Attack and Sideslip

\beq
\beqn
\alpha = tan^{-1}\left(\frac{w}{u}\right)\\
\beta = sin^{-1}\left(\frac{v}{V}\right)
\eeqn
\eeq

\noindent Lift Drag and Moment

\beq
\beqn
Lift~(L) = \frac{1}{2} \rho V^2 S C_L \\
Drag~(D) = \frac{1}{2} \rho V^2 S C_D\\
Roll~Moment~(L) = \frac{1}{2} \rho V^2 S b C_l\\
Pitch~Moment~(M) = \frac{1}{2} \rho V^2 S \bar{c}C_m\\
Yaw~Moment~(N) = \frac{1}{2} \rho V^2 S b C_n\\
\eeqn
\eeq

\noindent Lift and Drag Coefficients

\beq
\beqn
C_L = C_{L0} + C_{L\alpha}\alpha\\
C_L = C_{L\alpha}(\alpha-\alpha_0)\\
C_D = C_{D0} + C_{D\alpha}\alpha^2 \\
C_D = C_{D0} + k{C_L}^2 \\
\eeqn
\eeq

\noindent Non-dimensional Angular velocities

\beq
\beqn
\hat{p} = pb/2V\\
\hat{q} = q\bar{c}/2V\\
\hat{r} = rb/2V
\eeqn
\eeq

\noindent Pitch Moment equation

\beq
C_m = C_{m0} + C_{m\alpha}\alpha + C_{m\delta_e}\delta_e +
C_{mq}\hat{q}
\eeq

\beq
\beqn
C_{m0} = C_{MAC} + C_{L0}\bar{x}_{sm} \\
\bar{x}_{sm} = \frac{x_{cg}}{\bar{c}} - \frac{x_{acW}}{\bar{c}} \\
C_{m\alpha} = \left(C_{L\alpha,W} +
\frac{S_t}{S}C_{L\alpha,t}\right)\bar{x}_{sm} - V_HC_{L\alpha,t}\\
V_H = \frac{l_tS_t}{S\bar{c}} \\
C_{m\delta_e} = \left(C_{Lt\delta_e}\frac{S_t}{S}\right)\bar{x}_{sm}-V_HC_{Lt\delta_e}\\
C_{mq} = 2C_{L\alpha t}\frac{l^2}{\bar{c}^2}
\eeqn
\eeq

\noindent Max Lift to Drag Ratio (Only valid if ${C_{L0}=0}$)

\beq
\alpha_{max,L/D} = \sqrt{\frac{C_{D0}}{C_{D\alpha}}}
\eeq

\noindent Lift to Drag when $T=0$ (Sum of Forces still zero)

\beq
\beqn
\frac{D}{L} = tan(\alpha)\\ %%%Where does this come from? Sum of Forces = 0
Lcos(\alpha) + Dsin(\alpha) = W
\eeqn
\eeq

\noindent Airfoil and Wing Aerodynamics

\beq
\beqn
x_{ac} = c/4 & 
a = \frac{a_0}{1+\frac{a_0}{\pi e AR}} \\
AR = \frac{b^2}{S}
\eeqn
\eeq


\noindent Standard Atmosphere

\beq
\beqn
\rho = 1.225~kg/m^3 = 0.00238~slugs/ft^3\\
\mu_{\infty} = 1.81x10^{-5}~kg/(m-s)\\
a_{\infty} = 331.3~m/s
\eeqn
\eeq

\noindent General Notes

\begin{enumerate}
  \item In trim or steady and level or cruise $q=0$, $C_m=0$,$L=W$, $T=D$
  \item For symmetric airfoil $C_{MAC} = 0$ and
    $C_{L0}=0$ thus $C_{m0} = 0$
  \item For a flat plate all symmetric properties apply but
    $a_0 = 2\pi$
  \item Tail surfaces are always assumed to be flat plates
  \item For longitudinal problems, $\beta = 0$ so $v=0$ (side velocity)
\end{enumerate}

\end{multicols}


\end{document}
